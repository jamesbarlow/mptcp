Before looking at Multipath TCP with Coupled congestion control as commonly
deployed, we present results using Multipath TCP with New Reno congestion
control. Although this might seem unnecessary considering Multipath TCP is
rarely ever used with New Reno as it uses an unfair share of total available
bandwidth, we believe it is useful to show the subtle differences between
running two parallel wireless clients and one client with multiple wireless
interfaces. Using New Reno allows us to evaluate this separately from the
load balancing performed by Coupled.

\subsubsection{Downlink}
\label{sec:results-mptcp-down}

\begin{figure}[h]
 \centering
 \subfloat[][2.4 GHz, same channel] {\
% TODO: find good graph for this
%   \scalebox{0.33}{\input{graphs/sc-reno-down.tex}}\label{graph:sc-reno-down}
 }
 \subfloat[][2.4 GHz, disparate channel] {\
% TODO: find good graph for this
%   \scalebox{0.33}{\input{graphs/cc-reno-down.tex}}\label{graph:cc-reno-down}
 }
 \subfloat[][5 and 2.4 GHz] {\
% TODO: find good graph for this
%   \scalebox{0.33}{\input{graphs/cb-reno-down.tex}}\label{graph:cb-reno-down}
 }

 \caption{CDF of downlink throughputs using New Reno congestion control}\label{graph:reno-down}
\end{figure}

On the downlink, we are expecting each Multipath TCP subflow to get an equal
share of the network as the regular TCP on the same network, as there should be
no difference in the amount of intereference when running both WiFi interfaces
on the same machine compared to on separate machines when the interface is
barely transmitting (only ACKs). This is also what we see from the graphs in
figure~\ref{graph:reno-down}.

The per subflow fairness in the downlink case is achieved in two steps: firstly,
the wireless medium should be shared equally between the two APs (assuming they
both have data to send and their subbands are equally busy) thanks to carrier
sensing; second, the flows going through each AP should get an equal share of
that APs airtime because they are both running TCP New Reno on equal paths,
which should lead to them sharing the available bandwidth fairly.

\subsubsection{Uplink}
\label{sec:results-mptcp-up}

\begin{figure}[h]
 \centering
 \subfloat[][2.4 GHz, same channel] {\
   \scalebox{0.55}{\input{graphs/sc-reno-up.tex}}\label{graph:sc-reno-up}
 }
 \subfloat[][2.4 GHz, disparate channel] {\
   \scalebox{0.55}{\input{graphs/cc-reno-up.tex}}\label{graph:cc-reno-up}
 }
 \\
 \subfloat[][5 and 2.4 GHz] {\
   \scalebox{0.55}{\input{graphs/cb-reno-up.tex}}\label{graph:cb-reno-up}
 }

 % TODO: get more stable results for all of these?
 % TODO: confirm trend for 5 test
 \caption{CDF of uplink throughputs using New Reno congestion control}\label{graph:reno-up}
\end{figure}

Due to the volatile nature of wireless, it was hard to get consistent results
for these tests, but the general trend we observe is that on the uplink, the
story is much the same as on the downlink; each Multipath TCP subflow gets an
equal share of the total throughput to every other flow on that link as seen in
figure~\ref{graph:sc-reno-up} and figure~\ref{graph:sc-reno-up}. This also makes
sense conceptually as the 802.11 MAC provides fairness \textbf{per 802.11
client} using carrier sense. Note that this is different from the downlink case
where fairness is provided by carrier sense \textbf{and} the APs. Since
Multipath TCP has one client on every network, one would expect each subflow to
get half the available capacity on each network when using New Reno.

The equal distribution above gives Multipath TCP an aggregate throughput equal to
the sum of half of each link, and suggests that no additional interference
penalty is incurred by running both interfaces on the same client rather than on
different clients. We can also reason from this that Multipath TCP with Coupled
congestion control should theoretically be able to eventually reach the same
throughput as parallel clients if no other flows are competing on the network.

When the two interfaces are running on different bands, as in
figure~\ref{graph:cb-reno-up}, we see a curious effect where Multipath TCP backs
off the 2.4 GHz network somewhat. This was unexpected with New Reno as Multipath
TCP should not be trying to be fair to other flows, but rather try to maximize
the throughput on each path. It turns out that there is a connection between
flows when running Multipath TCP regardless of what congestion control is in
use. If Multipath TCP detects that a faster flow is forced to wait for ACKs on % TODO: We have to confirm that this is the case
the slower flow, it will… % TODO: finish this when confirmed

\subsubsection{A note on wireless experiments}
Running every test many times has proven very important throughout this project.
For the uplink reno 2.4 GHz tests for example, we initially suspected that
Multipath TCP was consistently seeing slightly lower throughput compared to
parallel, and we surmised that this might be because two WiFi interfaces
connected to one machine might somehow lead to more cross-interface interference
than with the same interfaces connected to different machines equally far apart.
After running more experiments to confirm, however, we quickly got contradicting
results where Multipath TCP was consistently faster than parallel. On balance,
we consider the results to pretty much even out across many tests. The slight
discrepancies of some links in those figures are simply due to WiFi interfaces
being extremely sensitive to their exact positioning, and average out over a
larger set of tests.
