Before discussing Multipath TCP with Coupled congestion control, we present
results using Multipath TCP with New Reno congestion control operating
independently for each subflow. This congestion control algorithm allows the
congestion window of each subflow to grow as if it were an independent TCP flow,
and is known to use an unfair share of the total available bandwidth [1]. These
results may therefore seem unnecessary, however we believe it is useful to show
the subtle difference between running two wireless clients in parallel and one
client with multiple wireless interfaces. New Reno allows us to evaluate this
separately from the load balancing performed by the Coupled algorithm.

% [1] Coupled Congestion Control RFC.

\subsubsection{Downlink}
\label{sec:results-mptcp-down}

\begin{figure}[h]
  \centering
  \subfloat[][2.4 GHz, same channel] {\
    \scalebox{0.70}{\input{graphs/sc-reno-down.tex}}\label{graph:sc-reno-down}
  }
  \\
  \subfloat[][2.4 GHz, non-overlapping channels] {\
    \scalebox{0.70}{\input{graphs/cc-reno-down.tex}}\label{graph:cc-reno-down}
  }
  \\
  \subfloat[][5 and 2.4 GHz] {\
    \scalebox{0.70}{\input{graphs/cb-reno-down.tex}}\label{graph:cb-reno-down}
  }
  \caption{Downlink, New Reno}\label{graph:reno-down}
\end{figure}

On the downlink, we would expect each Multipath TCP subflow to get an equal
share of the total throughput as a regular TCP flow, as there should be no
difference in interference when running both WiFi interfaces on the same machine
compared to on separate machines, when each interface is only transmitting ACKs.
This is shown in Figure~\ref{graph:reno-down}.

The per-subflow fairness in the downlink case is achieved in two steps. Firstly,
the wireless medium should be shared equally between the two APs (assuming they
both have data to send and their channels are equally busy) due to carrier
sense. Additionally, because TCP New Reno does not try to balance the load on
each flow, the sender would continuously try to push as much data as it can get
on the link and this is only limited by the finite buffer at the AP. The effect
of this is that the APs drop further incoming packets as soon as the buffers are
full and this notifies the sender to adapt its rate. However this does not mean
that TCP New Reno would reduce its aggressiveness, it just reduces the amount of
packets it has in flight at a given time. This value still grows as aggressively
as before hence the buffers at the AP are always full. The full buffers ensure
that the AP's always have data to send on behalf of each flow when it gets its
fair share of the medium as determined by carrier sense. Hence this should lead
to the flows sharing the available bandwidth fairly as they have roughly the
same amount of access to the medium.

\subsubsection{Uplink}
\label{sec:results-mptcp-up}

\begin{figure}[h]
  \centering
  \subfloat[][2.4 GHz, same channel] {\
    \scalebox{0.70}{\input{graphs/sc-reno-up.tex}}\label{graph:sc-reno-up}
  }
  \\
  \subfloat[][2.4 GHz, non-overlapping channels] {\
    \scalebox{0.70}{\input{graphs/cc-reno-up.tex}}\label{graph:cc-reno-up}
  }
  \\
  \subfloat[][5 and 2.4 GHz] {\
    \scalebox{0.70}{\input{graphs/cb-reno-up.tex}}\label{graph:cb-reno-up}
  }
  \caption{Uplink, New Reno}\label{graph:reno-up}
\end{figure}

Due to the unpredictable nature of wireless and varying levels of background
traffic, obtaining consistent results for these tests was a time consuming
process. The general trend we observe in Figure~\ref{graph:reno-up} is the
same seen on the downlink; each Multipath TCP subflow gets an equal
share of the total throughput to every other flow on that link. This also makes
sense conceptually, as in the uplink case the 802.11 MAC provides fairness
\textbf{per 802.11 client} using carrier sense, and in the downlink case
fairness is provided by carrier sense \textbf{and} the APs. Since Multipath TCP
acts as one client on each network, one would expect each subflow to get half
the available capacity on each network when using New Reno.

The equal distribution above gives Multipath TCP an aggregate throughput equal
to the sum of half of each link, and suggests that no additional interference
penalty is incurred by running both interfaces on the same client rather than on
different clients. We can also reason from this that Multipath TCP with Coupled
congestion control should theoretically be able to eventually reach the same
throughput as parallel clients if no other flows are competing on the network; 
this would be looked at in \S\ref{sec:results-performance}.

When the two interfaces are running on different bands, as in
Figure~\ref{graph:cb-reno-up}, we see a curious effect where Multipath TCP
lowers its send rate on the 2.4 GHz network. This was unexpected with New Reno
as Multipath TCP should not be trying to be fair to other flows, but rather try
to maximize the throughput on each path. It turns out that there is a connection
between flows when running Multipath TCP regardless of what congestion control
is in use. If Multipath TCP detects that a faster flow is forced to wait for
ACKs on the slower flow, it will…
[the graph refernced above doesnt actaully show lower send rate on 2.4GHz
do we have a graph that shows this?? consider removing the paragraph ]
% TODO: We have to confirm that this is the case
% TODO: finish this when confirmed
% vim:textwidth=80:colorcolumn=80:spell:
