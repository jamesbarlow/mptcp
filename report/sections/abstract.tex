This report evaluates the effect of WiFi interference on Multipath TCP. 
The results presented in this report show that there can be a performance gain 
in using Multipath TCP with WiFi both on the downlink and uplink.

In both cases, they show that on two idle path it can attain an aggregate
throughput which is equal to the sum of both links or more. However with 
contending traffic results for the downlink show that the aggregate throughput 
is reduced to the sum of the best single link. The uplink results show that the 
aggregate throughput is more than that of the best link and in some cases equal 
to the sum of both links.

The behaviour of Multipath TCP on the uplink is unfair to other flows on the 
medium as it gets a larger share in total. This contradicts the second rule of 
its default congestion control algorithm; the Coupled congestion control which 
is do no harm. 
