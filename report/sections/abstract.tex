In this report we evaluate the behaviour of Multipath TCP when using multiple
WiFi interfaces on a single host; especially with regards to how
self-interference affects its ability to provide reliability and improved
throughput. We also explore other differences between wired and wireless
networks, in particular how wireless networks may cause Multipath TCP to not
behave fairly to competing flows, even with Coupled congestion control.

Our results show that Multipath TCP can improve both reliability and throughput
when using multiple WiFi links for both downlink and uplink traffic. In both
cases, we show that on two idle wireless networks, Multipath TCP can achieve an
aggregate throughput equal to the sum of both links. We also demonstrate how
Multipath TCP behaves unfairly on the uplink when competing flows are present on
any of the wireless networks.
% vim:textwidth=80:colorcolumn=80:spell:
